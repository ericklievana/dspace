\documentclass[a4paper]{article}

\usepackage[margin=1in]{geometry}
\usepackage[utf8]{inputenc}
\usepackage[spanish]{babel}
\usepackage{graphicx}
\graphicspath{{./images/}}

\setlength\parindent{0pt}


\begin{document}
\begin{titlepage}
  \begin{center}
    \begin{figure}[h]
      \centering
      \includegraphics[width=0.3\linewidth]{buaplogo.png}
    \end{figure}
    \Huge
    Benemérita Universidad Autónoma de Puebla

    \begin{figure}[h]
      \centering
      \includegraphics[width=0.3\linewidth]{fcclogo.png}
    \end{figure}
    \huge
    Facultad de Ciencias de la Computación

    \vspace{1em}
    Administración de Redes

    \Huge
    \vspace{1em}
    \textbf{Cuestionario}
    \vspace{1em}

    \large
    \begin{tabular}{ l l }
      García Rojas Alan & 201765944\\
      Liévana Poy Erick & 201742006\\
      Lima Estrada Efraín & 201705754\\
      Reyes Reyes Julián & 201643331\\
    \end{tabular}
    \vspace{1em}

    Prof. Larios Gómez Mariano

    \vfill
    Otoño 2021
  \end{center}
\end{titlepage}

\begin{enumerate}
  \item ¿Qué es DSpace?
    \begin{enumerate}
      \item Es un software para la construcción de repositorios digitales abiertos.
      \item Es un software para desconstuccion de repositorios.
      \item Es un software para el desarrollo de aplicaciones
    \end{enumerate}

  \item En qué sistemas operativos se puede instalar DSpace?
    \begin{enumerate}
      \item Mac OSX
      \item Linux
      \item Windows
      \item Todas las anteriores
    \end{enumerate}

  \item ¿Con qué bases de datos funciona DSpace?
    \begin{enumerate}
      \item DB2 y Oracle
      \item PostgreSQL y Oracle DataBase
      \item DB2 y SQL
    \end{enumerate}

  \item ¿Qué significa frontend?
    \begin{enumerate}
      \item API del servidor
      \item Diseño logico
      \item Interfaz de usuario
    \end{enumerate}

  \item ¿Qué significa backend?
    \begin{enumerate}
      \item API del servidor
      \item Diseño logico
      \item Interfaz de usuario
    \end{enumerate}

  \item ¿Qué dependencias del backend son necesarias para la instalación DSpace?
    \begin{enumerate}
      \item Java JDK
      \item Apache Maven
      \item Apache Ant
      \item base de datos
      \item Apache Solr
      \item Apache Tomcat
      \item Todas las anteriores
    \end{enumerate}

  \item ¿Qué dependencias del frontend son necesarias para la instalación DSpace?
    \begin{enumerate}
      \item PM2
      \item Yarn
      \item Node.js, Yarn, PM2
    \end{enumerate}

  \item ¿Qué extensión es necesaria activar cuando se instala la base de datos?
    \begin{enumerate}
      \item Login.database
      \item Pgcrypto
      \item Db.begin
    \end{enumerate}

  \item ¿Cual es la visión de DSpace?
    \begin{enumerate}
      \item Producir una opción mundial de repositorios
      \item Trabajar en empresas pequeñas
      \item Dar abasto con su software
    \end{enumerate}

  \item ¿Cual es la misión de DSpace?
    \begin{enumerate}
      \item Basar los repositorios en una pagina Web
      \item Crear un software de código abierto a través de una comunidad de desarrolladores activa
      \item Buscar software libre
    \end{enumerate}

  \item Seleccione 3 formatos más comunes que se administran en el entorno de DSpace
    \begin{enumerate}
      \item Word, PDF, JPEG
      \item MPEG, TIFF, PNG
      \item PNG y PDF
    \end{enumerate}

  \item ¿Qué significa un bitstream?
    \begin{enumerate}
      \item Es un archivo de datos
      \item Datos del sistema
      \item Un archivo de bits
    \end{enumerate}

  \item ¿Qué hace la extensión pgcrypto?
    \begin{enumerate}
      \item Inicia la base de datos
      \item Crea un repositorio
      \item Permite a DSpace crear identificadores universalmente únicos.
    \end{enumerate}

  \item ¿Qué permite la creación de repositorios multidisciplinarios?
    \begin{enumerate}
      \item La arquitectura modular
      \item La arquitectura circular
      \item La arquitectura simple
    \end{enumerate}

  \item Seleccione 3 estándares que cumplan los protocolos de acceso de DSpace
    \begin{enumerate}
      \item OAI-PMH, OAI-ORE, SWORD OpenURL
      \item WebDAV, OpenSearch, PNG
      \item PNG, RSS, ATOM.
    \end{enumerate}

  \item ¿Cual es el uso más común de DSpace como repositorio?
    \begin{enumerate}
      \item Repositorio institucional
      \item Repositorio empresarial
      \item Repositorio personal
    \end{enumerate}

  \item ¿Que usos tiene DSpace como repositorio?
    \begin{enumerate}
      \item Repositorio basado en temas
      \item Repositorio de conjunto de datos
      \item Repositorio en medios
      \item Todas las anteriores
    \end{enumerate}

  \item ¿Qué significa que DSpace sea de código abierto BSD?
    \begin{enumerate}
      \item Que cualquiera puede desarrollar sofwtare
      \item Que cualquiera crea su organizacion
      \item Que cualquier organización puede modificar e integrar código.
    \end{enumerate}

  \item ¿Qué costo tiene DSpace?
    \begin{enumerate}
      \item 100 dolares
      \item 10 dolares
      \item 1 dolar
      \item Ninguno, es gratuito.
    \end{enumerate}

  \item ¿En qué formas clave DSpace permite la personalización?
    \begin{enumerate}
      \item En la interfaz de usuario, metadatos y en configuración de búsqueda y exploración
      \item la creación de software
      \item En la busqueda de contenido
    \end{enumerate}

  \item Seleccione las características de Dspace
    \begin{enumerate}
      \item Acceso facil y abierto
      \item Amplia comunidad de desarrolladores
      \item Mejora continua de software
      \item Todas las anteriores
    \end{enumerate}

  \item ¿En que sentido esta diseñado Dspace?
    \begin{enumerate}
      \item Para manejar cualquier tipo de formato
      \item Para el diseño Web
      \item Para el manejo de software libre
    \end{enumerate}

  \item ¿Cuál es la forma de sumisión de Dspace?
    \begin{enumerate}
      \item Interfaz web que crea un item al depositar un archivo
      \item La clave de los repositorios
      \item Creacion de software
    \end{enumerate}

  \item ¿Cómo se organizan los bitstream?
    \begin{enumerate}
      \item En conjunto no relacionados
      \item En conjuntos relacionados
      \item De manera aleatoria
    \end{enumerate}

  \item ¿En qué año se creo Dspace?
    \begin{enumerate}
      \item 2002
      \item 1998
      \item 2000
    \end{enumerate}

  \item ¿Qué es un item?
    \begin{enumerate}
      \item Archivo atómico
      \item Archivo de video
      \item Archivo para imágenes
    \end{enumerate}

  \item ¿Qué expone un item?
    \begin{enumerate}
      \item Archivos de datos
      \item Metadatos
      \item Imágenes de archivos
    \end{enumerate}

  \item ¿Cómo se organizan los items?
    \begin{enumerate}
      \item Por el material logicamente relacionado
      \item Por nombre
      \item De manera aleatoria
    \end{enumerate}

  \item ¿Qué universidad diseño DSpace?
    \begin{enumerate}
      \item MIT
      \item Harvard
      \item Oxford
    \end{enumerate}

  \item ¿Qué laboratorios ayudo a la creación de Dspace?
    \begin{enumerate}
      \item Microsoft
      \item Linux
      \item HP
    \end{enumerate}
\end{enumerate}
\end{document}
